\section{Введение}

Распознавание именованных сущностей (Named entity recognition, NER) это одна из задач обработки естественного языка; задача обнаружения и классификации слов в тексте на несколько заранее определённых категорий, таких как, например, люди, места, организации и т.д. Распознавание именованных сущностей имеет множество применений, используется в автоматическом разделении на категории текстов, рекомендательных системах, системах извлечения информации. Как задача распознавание именованных сущностей была сформулирована ещё в прошлом веке, однако широкое распространение получила только в последнем десятилетии. Развитие глубоких нейронных сетей дало значительный толчок развитию обработке естественных языков, и, как следствие, задаче распознавания именованных сущностей. Были изобретены более эффективные и точные модели, которые показывают хорошие результаты. Однако существуют и серьёзные проблемы, связанные с данной задачей. Во-первых, упомянутые выше модели с хорошими результатами существуют только для широко распространённых языков, для которых имеются размеченные корпусы, а языки, которые не входят в <<топ-10>> по числу носителей, оказываются за бортом. Во-вторых, у компаний и исследователей нет причины вкладываться в задачу распознавания именованных сущностей для языков с малыми ресурами, так как, скорее всего, это не сможет принести большой выгоды в дальнейшем из-за сравнительно небольшого числа носителей (как и было сказано ранее). В то же время у меня есть причина: татарский язык является родным языком для меня, и я стараюсь сохранять его и продвигать его значимость, в том числе и с помощью такой работы.

Помимо патриотических мотивов существуют и мотивы прагматические: развитие распознавание именованных сущностей для татарского языка может быть использовано для всей кыпчакской группы тюркской ветви языков (татарский, башкирский, карачаевыо-балкарский, казахский, киргизский и др.). Это связано с тем, что языки тюркской ветви достаточно похожи между собой, как грамматически, так и лексически, как следствие, решение задачи для одного языка скорее всего будет иметь неплохие шансы и для других языков данной группы. К сожалению, это не сработает для турецкого языка, во-первых, потому что он относится к огузской группе, во-вторых (и это главная причина): там используется другой алфавит. Тюркская ветвь включает себя языки с различной письменностью, что усложняет возможность экстраполяции модели на <<похожие>> языки.

Также на тему конкретно татарского языка существует работа \cite{Nevzorova} исследователей из Академии наук Республики Татарстан. Далее я более подробно рассмотрю их работу в своем исследовании.

Целью работы является получить размеченный корпус и обученную модель, распознающую именованные сущности и сравнить полученные результаты с текущими имеющимися в этом поле. Были сделаны обзор литературы, получение и разметка данных, обучение текущих лучших моделей. 

\textcolor{red}{TODO} Помогите я не знаю как написать о своих целях и задачах.