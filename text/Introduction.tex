\section{Введение}

Извлечение именованных сущностей (Named entity recognition, NER) это одна из задач обработки естественного языка. Словосочетания, однозначно определяющие некоторого человека, организацию, географический или другой объект, называются именованными сущностями. Задача обнаружения и классификации таких словосочетаний в тексте на несколько заранее определённых категорий и есть задача извлечения именованных сущностей. На вход подаётся текст (предложение), на выходе --- массив из меток для каждой словоформы (словоформы -- слова, знаки препинания, числа которые есть в тексте).

\begin{table}[h]
\begin{tabular}[h]{ccccccccccc}
\textcolor{green}{B-PER} & \textcolor{green}{I-PER} & O & O & \textcolor{blue}{B-ORG} &  \textcolor{blue}{I-ORG} &  \textcolor{blue}{I-ORG} & O & \textcolor{red}{B-TIM} & O & O \\
Иван & Петров & преподает & в & Высшей & Школе & Экономики & с & 2014 & года & . \\
\end{tabular}
\caption{Пример размеченных данных, использована нотация BIO}
\end{table}

Извлечение именованных сущностей имеет множество применений: в автоматическом разделении на категории текстов, в рекомендательных системах, в системах извлечения информации. Как задача извлечение именованных сущностей была сформулирована ещё в 1996 году \cite{first_NER}, однако широкое распространение получила только в последнем десятилетии \cite{DBLP:journals/corr/abs-1812-09449}. Развитие глубоких нейронных сетей дало значительный толчок развитию обработки естественных языков, и, как следствие, задаче извлечения именованных сущностей. Были изобретены более эффективные и точные модели, которые показывают хорошие результаты. Однако остаются и нерешенные проблемы, связанные с данной задачей. Во-первых, упомянутые выше модели с хорошими результатами существуют только для широко распространённых языков, для которых имеются размеченные корпусы, а языки, не входящие в <<топ-10>> по числу носителей, оказываются вне внимания исследователей. Во-вторых, у компаний и исследователей недостаточно причин прикладывать усилия к задаче извлечения именованных сущностей для языков с малыми ресурсами, так как, скорее всего, это не сможет принести большой выгоды в дальнейшем из-за сравнительно небольшого числа носителей. У меня есть причина личного характера: татарский язык является родным языком для меня, и я стараюсь сохранять и развивать его, в том числе и с помощью этой работы.

Помимо патриотических мотивов есть и прагматические мотивы: решение задачи извлечения именованных сущностей для татарского языка может быть использовано для всей кыпчакской группы тюркской ветви языков (татарский, башкирский, карачаево-балкарский, казахский, киргизский и др.). Это связано с тем, что языки тюркской ветви достаточно похожи между собой, как грамматически, так и лексически. Как следствие, решение задачи для одного языка может быть и для других языков данной группы. 

В тюркской языковой семье языком с наиболее богатыми ресурсами является турецкий язык, поскольку он является национальным языком. Остаётся открытым вопрос, насколько ресурсы турецкого языка могли бы помочь для решения задачи извлечения именованных сущностей на татарском языке. В данной работе я отказалась от данного подхода по следующим причинам:
\begin{itemize}
\item турецкий язык относится к другой группе языков: огузской;
\item турецкий язык использует латинский алфавит, в то время как татарский язык преимущественно записывается на кириллице
\end{itemize}

Работа \cite{Nevzorova} исследователей из Академии наук Республики Татарстан даёт рекомендации к разметке корпуса. Далее я более подробно рассмотрю их работу в своем исследовании.

Целью работы является получить размеченный корпус и обученную модель, распознающую именованные сущности и сравнить полученные результаты с ранее имеющимися в этом поле. Работа содержит в себе обзор литературы, получение и разметку данных, выбор двух лучших известных моделей, обучение моделей и экспериментальную оценку. Рассматриваемые модели решают задачу, сформулированную следующим образом: на вход модели подаётся произвольный текст (предложение), а на выходе получается текст с тегами для каждой именованной сущности, встречающейся в тексте.












