\documentclass[a4paper,12pt]{article}
\usepackage{header}
\usepackage{enumitem}
\usepackage{subcaption}

\providecommand{\keywords}[1]
{
  \small	
  \textbf{\textit{Keywords---}} #1
}

\begin{document}
\selectlanguage{russian}

    
\large
\bigskip

\begin{center}
{\large Федеральное государственное автономное образовательное 

учреждение высшего образования

Национальный исследовательский университет

\smallskip

<<Высшая школа экономики>>

}

\bigskip
\bigskip
\bigskip

{ \large
Факультет компьютерных наук

Основная образовательная программа

Прикладная математика и информатика
}
\end{center}

\begin{center}
  {\large Выпускная квалификационная работа}
  
  {\large на тему}
\end{center}

\begin{center}
  \textbf{\huge Распознавание именованных сущностей для языков с малыми ресурсами}
\end{center}


\renewcommand{\arraystretch}{1.8} %% increase table row spacing

\bigskip
\bigskip
\bigskip
\bigskip

{\large Выполнила студентка группы БПМИ151, 4 курса,

\hspace{3cm} Закирова Ксения Игоревна


Научный руководитель:

\hspace{3cm} Доцент, кандидат технических наук,

\hspace{3cm} Артемова Екатерина Леонидовна


}
    
\bigskip
\bigskip

\bigskip
\bigskip
\bigskip
\pagestyle{empty}

\vfill
\begin{center}
  {Москва, 2020}
\end{center}

\newpage
    \pagestyle{empty}
    \tableofcontents
    \pagestyle{plain}
    
\newpage    
    
\selectlanguage{english}
\begin{abstract}
In this work, I investigate the approaches to the problem of named entity recognition in the Tatar language. The Tatar language is low-resource, so I tackled both initial data collection and modelling. I automatically annotated corpora Tugan Tel and Wikipedia and I present a list of named entities. Corpora contain labels such as PER, LOC, ORG and MISC in BIO notation. I trained BiLSTM-CRF and BERT models. The BERT-based model achieves 0.47 average F-score. 
\end{abstract} \hspace{10pt}

%TC:ignore
\keywords{Named Entity Recognition, NER, Tatar language, low-resource languages}
%TC:endignore
\selectlanguage{russian}

\begin{abstract}
В данной работе я рассмотрела задачу извлечения именованных сущностей в татарском языке, собрала данные для корпуса Википедии и обучила машинную модель. Татарский является малоресурсным языком, для которого нет доступных решений в литературе. Результатом работы является список именованных сущностей и размеченный корпус на основе Википедии, который я предоставляю в открытый доступ. Корпус содержат теги PER, LOC, ORG и MISC в нотации BIO. Я обучила модели BiLSTM-CRF и BERT. BERT показал средний результат метрики f-score 0.47 на тестовом наборе, который был размечен вручную.
\end{abstract} 
%TC:ignore
\keywords{Извлечение именованных сущностей, татарский язык, малоресурсные языки}
%TC:endignore

\hspace{10pt}

    \section{Введение}

Распознавание именованных сущностей (Named entity recognition, NER) это одна из задач обработки естественного языка; задача обнаружения и классификации слов в тексте на несколько заранее определённых категорий, таких как, имена, люди, географические названия, организации и т.д. Распознавание именованных сущностей имеет множество применений в автоматическом разделении на категории текстов, рекомендательных системах, системах извлечения информации. Как задача распознавание именованных сущностей была сформулирована ещё в 1996 году \cite{first_NER}, однако широкое распространение получила только в последнем десятилетии \cite{DBLP:journals/corr/abs-1812-09449}. Развитие глубоких нейронных сетей дало значительный толчок развитию обработке естественных языков, и, как следствие, задаче распознавания именованных сущностей. Были изобретены более эффективные и точные модели, которые показывают хорошие результаты. Однако остаются и нерешенные проблемы, связанные с данной задачей. Во-первых, упомянутые выше модели с хорошими результатами существуют только для широко распространённых языков, для которых имеются размеченные корпусы, а языки, которые не входят в <<топ-10>> по числу носителей, оказываются вне внимания исследователей. Во-вторых, у компаний и исследователей недостаточно причин прикладывать усилия к задаче распознавания именованных сущностей для языков с малыми ресурсами, так как, скорее всего, это не сможет принести большой выгоды в дальнейшем из-за сравнительно небольшого числа носителей. У меня есть причина личного характера: татарский язык является родным языком для меня, и я стараюсь сохранять и развивать, в том числе и с помощью этой работы.

Помимо патриотических мотивов есть и прагматические мотивы: развитие распознавание именованных сущностей для татарского языка может быть использовано для всей кыпчакской группы тюркской ветви языков (татарский, башкирский, карачаевыо-балкарский, казахский, киргизский и др.). Это связано с тем, что языки тюркской ветви достаточно похожи между собой, как грамматически, так и лексически. Как следствие, решение задачи для одного языка скорее всего будет иметь неплохие шансы и для других языков данной группы. К сожалению, это не сработает для турецкого языка, во-первых, потому что он относится к огузской группе, во-вторых (и это главная причина): там используется другой алфавит. Тюркская ветвь включает себя языки с различной письменностью, что усложняет возможность экстраполяции модели на <<похожие>> языки. %TODO найди ссылку!!!

Работа \cite{Nevzorova} исследователей из Академии наук Республики Татарстан даёт рекомендации к разметки корпуса. Далее я более подробно рассмотрю их работу в своем исследовании.

Целью работы является получить размеченный корпус и обученную модель, распознающую именованные сущности и сравнить полученные результаты с ранее имеющимися имеющимися в этом поле. Работа содержит в себе обзор литературы, получение и разметку данных, выбор двух текущих лучших моделей, обучение моделей и экспериментальную оценку.

    \section{Обзор литературы}

На данный момент существует одна релевантная моему исследованию статья про работу конкретно в татарском языке, однако она не содержит в себе использование методов современного машинного обучения. Также есть работы на темы других языков с малыми ресурсами и работы о моделях, которые были полезны в моей работе.

\subsection{Named Entity Recognition in Tatar: Corpus-Based Algorithm}

Самая близкая к моей работе это статья <<Named Entity Recognition in Tatar:
Corpus-Based Algorithm>> от О. Невзоровой, Д. Мухамедшина и А. Галиевой, Академия наук Республики Татарстан. В статье они рассказывают, как разметили корпус <<Туган Тел>>[ссылка], использовав следующие категории: книги, рестораны, фильмы, журналы, компании, аэропорты, корпорации, языки, колледжи, университеты, школы, магазины, музеи и больницы. Несмотря на наличие в названии распознавания именованных сущностей, они скорее использовали полуручной метод разметки. 

TODO сделай везде доллары, где есть математика!!!

TODO Тут будет описание их статьи.

TODO Куда вставить описание Туган Тел?

\subsubsection{Использованные данные}

Туган Тел -- это корпус текстов на татарском языке, разработанный Институтом прикладной
семиотики Академии наук Республики Татарстан. Корпус предназначен для широкого круга 
пользователей: лингвистов, специалистов в татарском языке, преподавателей татарского и всем 
тем, кому может понадобиться набор текстов на татарском языке. Основными функциями корпуса 
являются: поиск  по словоформе, лемме (лексеме), набору морфологических параметров. 
Существует система <<корпус-менеджер>>, которая поддерживает данные функции.  На данный 
момент существует проект разработки электронного корпуса, который также включает в себя 
автоматическую разметку корпуса, чем и занималась команда Невзоровой. Корпус включает в 
себя татарские тексты различных жанров, такие как художественная литература, тексты СМИ, 
тексты официальных документов, учебная литература, научные публикации и др. Каждый
документ имеет метаописание, включающее в себя автора и его пол, выходные данные, дату 
создания, жанр, части, главы и др. Тексты, включенные в корпус, снабжены автоматической 
морфологической разметкой, которая включает в себя информацию о части речи и 
грамматической характеристики словоформы. Морфологическая разметка текстов корпуса 
выполняется автоматически с использованием модуля двухуровневого морфологического анализа 
татарского языка, реализованного в программном инструментарии PC-KIMMO, с чем связан ряд 
проблем в использовании данного корпуса, о которых я скажу в основной части работы. На 
декабрь 2019 года в корпусе 194 млн. словоформ. 

TODO Нужен ли этот абзац или нафиг его надо?

В качестве релевантных статей Невзорова at al указывают LingPipe[ссылка], команда которой 
решает похожую задачу в английском языке (TODO проверить, так ли это, и о чём вообще статья), 
Annie[ссылка], Afner[ссылка], ссылаются также на марковские цепи, решающие деревья и CRF,
которые потом не используют (в то время как я в этой работе использую). В общем, много хороших 
разных ссылочек, которые надо изучить подробнее, чтобы что-нибудь про них написать. Или вырезать это всё в целом.

\subsubsection{Разбор алгоритма, предложенного в статье:}

Представленный алгоритм основан на идее сравнения $n$-грамм. Сравнение происходит на всём 
объёме корпуса, что увеличивает точность результата, заявляют авторы статьи. Алгоритм является итеративным, причём количество итераций определяется пользователем (что показывает, что их алгоритм является в некоторой степени полуручным.

Первым шагом алгоритма включает в себя выборку по поисковому запросу. Запрос может 
представлять собой форму слова, лемму, фразу или поиск по морфологическим параметром. 
Выборка представляет собой набор биграмм и их количество вхождений в текст. В биграмме одно 
слово является запросом, в то время как второе слово может добавляться слева или справа, данный 
параметр выбирается пользователем. Полученный список биграмм отсортировывается по частоте 
вхождений в корпус и в выборке остаются только самые частотные (например, первые 95\%, в статье 
этот параметр обычно был равен 80\%). Порог отсечения (в статье он называется <<индекс 
покрытия>>, <<covering index>>) более частотных вхождений также выбирается пользователем. 
Урезанный по порогу список биграмм используется как входные данные для второй итерации 
алгоритма: каждая биграмма ищется по корпусу как фраза и, аналогично первому шагу, 
составляются триграммы и их частоты. Точно так же выбираются самые частотные триграммы 
(третье слово может добавляться справа или слева), список обрезается по пороговому значению и,
при желании, алгоритм продолжается дальше, используя на вход уже список триграмм.

Таким образом алгоритм использует $n$-граммы для поиска $(n+1)$-грамм, некоторые из которых будут отсечены порогом, а остальные использованы в следующем шаге алгоритма.

\subsubsection{Окончание алгоритма:}

Существует такое понятие как <<точность сравнения>> (<<accuracy of matching>>) $P$, которое задаётся пользователем в процентах. Если частота $n$-граммы меньше $P$ от количества найденных $(n+1)$-грамм, то $n$-грамма считается именованной сущностью, иначе алгоритм переходит на следующую итерацию. Таким образом, в финальный результат входят самые стабильные $n$-граммы разной длины, включая результаты поиска изначального поискового запроса.

Запрос извлечения именованных сущностей представляет собой кортеж (1), где $Q_1$ и $Q_2$ никак не объясняются, $L, R$ это, соответственно, порог ограничения итераций добавления слов слева и справа, $C$ --- порог отсечения частотности на каждой итерации (covering index), $P$ --- порог для принятия решения о включении фразы в итоговый список именованных сущностей (accuracy of matching). В качестве примера они снова ссылаются на формулу (1) (скорее всего, имелась в виду формула (2) из примера).


\[Q = (Q_1, Q_2, L, R, C, P)\]

\subsubsection{Эксперименты:}

Тут, конечно, всё хитро: выставляются, естественно, только те результаты, где всё получилось хорошо, а где получилось не слишком хорошо --- об этом ничего не сказано. Исследователи перечисляют довольно много категорий, над которыми они экспериментировали, но результаты они показали на словах <<министерство>>, <<улица>>, <<язык>>, <<ресторан>> и <<корпорация>>. Одной из очевидных дополнительных тем являлись бы <<реки>>, но Невзорова at al. на реках экспериментировать не стали.

Также в данной статье очень интересный способ оценки результатов. Стандартные accuracy, precision и recall (и производная от них F-score) в статье не упоминается, к сожалению, но по тексту можно вычленить нечто на них похожее. 

TODO Допиши авторов!!!

\subsection{Low-Resource Named Entity Recognition with Cross-Lingual, Character-Level Neural Conditional Random Fields}

Что-нибудь тут напишу о том, что статья хорошая, использовала я похожую архитектуру, но не их.

\subsection{A Neural Layered Model for Nested Named Entity Recognition}

Хорошая статья, из которой я решила использовать модель. 

\subsection{Datasets and Baselines for Named Entity Recognition in Armenian Texts}

Очень вдохновляющая статья (вообще говоря, магистерская работа), которая, по факту, и стала решающей при выборе темы. Тема моей работы очень близка к теме работы данных исследователей, за исключением языка: у них, как понятно из названия, армянский язык, который так же относится к языкам малой языковой группы.

В отличие от моего случая, где существует релевантная работа, поднимавшая раньше тему моей работы, Т. Гукасян, Г. Давтян, К. Аветисян и И. Андрианов стали, можно сказать, первопроходцами в своей области, поскольку никто не делал подобных работ для армянского языка. У них не было подобранного и размеченного корпуса текста, поэтому, помимо распознавания именованных сущностей, они занимались также и сбором и разметкой данных. Их модель включала в себя CRF, которую я использую и в своей работе, и рекомендую как хорошую модель для языков с малыми ресурсами.

В своей работе исследователи не использовали BERT, поскольку это относительно новая модель, а статья вышла в конце 2018 года. У меня, к счастью, такая возможность есть, поэтому я ей воспользовалась.











    \section{Методология}

Целью работы было получить размеченный корпус и обученную модель, распознающую именованные сущности. После обзора литературы были намечены задачи и работа была предварительно разделена на несколько этапов.

\begin{enumerate}
\item Получение и разметка данных
\item Обучение и тюнинг моделей
\item Сравнение результатов
\end{enumerate}

Но в течение работы по нескольким причинам были внесены корректировки. Во-первых, как я упоминала ранее в обзоре литературы, с представленными результатами в статье Невзоровой невозможно сравниваться, поскольку цели моей и их работ различаются. Во-вторых, качество полученных данных оказалось не лучшим из возможных, а алгоритм Невзоровой, разработанный как раз для разметки данных, мог бы улучшить имеющийся корпус, используемый для обучения моделей. Как следствие, было принято решение воспроизвести алгоритм из статьи Невзоровой насколько это возможно и воспользоваться полученными результатами.

\begin{enumerate}
\item Получение и разметка данных
\item Обучение и тюнинг моделей
\item Воспроизведение статьи Невзоровой
\item Разметка данных с помощью алгоритма Невзоровой
\item Обучение и тюнинг моделей
\item Сравнение результатов
\end{enumerate}


\subsection{Получение и разметка данных}

Обзор литературы показал, что существует корпус татарских текстов Туган Тел\cite{tugan_tel}. Данный копрус имеет также свою систему <<корпус-менеджер>>, которая представлена в виде сайта. На этом сайте можно искать по словоформе или лемме с огромным количеством параметров [\ref{fig:tugan_tel_1}], однако возможности просто скачать весь корпус не оказалось. Я предполагаю, что у Академии наук Республики Татарстан есть API для исполнения запросов на большом количестве данных и в каком-то более удобном формате, чем запрос на сайте, но у меня доступа к такому ресурсу нет. 

\begin{figure}
\caption{Параметры на сайте \href{http://tugantel.tatar/}{tugantel.tatar} для поиска по корпусу}
\includegraphics[width=\textwidth]{pics/tugan_tel_1}
\label{fig:tugan_tel_1}
\end{figure}


Я связалась с Невзоровой по указанной в статье электронной почте, чтобы узнать подробности об их работе и попросить о сотрудничестве. Невзорова ответила на моё письмо и предоставила мне доступ к корпусу.

Корпус представляет из себя $.zip$ файл, состоящий из $7557$ $.txt$ файлов, в общей сложности весом $1\ 183\ 023\ 978$ Б. Как уже упоминалось ранее, корпус Туган Тел автоматически размечен с помощью программного инструментарии PC-KIMMO. Разметка выглядит следующим образом: \ref{table:sample_sent} \ref{fig:sample_sent}


\begin{figure}
\caption{Пример случайного предложения из корпуса Туган Тел}
\includegraphics[width=\textwidth]{pics/sample_sent}
\label{fig:sample_sent}
\end{figure}

\begin{table}[h!]
\centering
\begin{tabular}[h]{| l | l |}
\hline
Аның & аны+PN+POSS\_2SG(Ың)+Nom;аның+PN;ул+PN+GEN(нЫң); \\
\hline
дөньяга & дөнья+N+Sg+DIR(ГА); \\
\hline
күз & күз+N+Sg+Nom; \\
\hline
карашы & караш+N+Sg+POSS\_3(СЫ)+Nom; \\
\hline
хаман &  хаман+Adv; \\
\hline
үзгәрми & үзгәр+V+NEG(мА)+PRES(Й); \\
\hline
.  & Type1 \\
\hline
\end{tabular}

Перевод: Его мировоззрение постоянно меняется.
\caption{Пример случайного предложения из корпуса Туган Тел}
\label{table:sample_sent}
\end{table}


Первое слово в каждом файле распознано как Error, все русские слова не распознаны (а их в татарском языке некоторое ненулевое количество, так как происходит какое-то достаточное количество заимствований). К сожалению для меня, очень часто русскоязычные слова оказывались как раз именованными сущностями, такими как, например, названия улиц, но распознаны они были как Error, что очень печально.

*TODO: написать \% Error от общего количества слов и привести пару примеров*

В этом файле огромная куча всяких тегов, про которые без бутылки не разберешься (и даже гугление этого парсера, блин, не помогает). Но эмпирическим методом было выяснено, что атрибут PROP --- это как раз та самая именованная сущность, что нам нужна. На основании этого все остальные атрибуты были выброшены, а PROP заменен на B-PER, так как используемая модель использовала IOB метод разметки (TODO: написать про IOB метод разметки).

Всего в текстах 30 753 824 слов, из них 534 514 это автоматически размеченные именованные сущности, что составляет $1,738\%$ от всех слов. 

*TODO: привести примеры слов с атрибутом PROP*.

*TODO: подсчитать, раз в сколько предложений в среднем встречается именованная сущность*

Также в качестве корпуса текста есть татарская википедия. На данный момент она содержит 89 252 статей, причём некоторые из них сгенерированы автоматически, что ухудшает качество текстов как корпуса для обучения, так как некоторые фразы становятся частотными не из-за того, что они действительно часто используются в языке, а из-за множества сгенерированных статей. *TODO вставить пример про бассейны*

Проблема с татарской википедией также была в смеси латиницы и кириллицы *TODO экскурс в историю по поводу того, как мы до такой жизни дошли*, так что пришлось ещё и из латиницы в кириллицу переводить.


\subsection{Обучение и тюнинг моделей}

\subsubsection{BiLSTM-CRF}

Была использована модель BiLSTM-CRF, которая норм заработала. Она использует разметку IOB, так что данные пришлось немного подкорректировать и добавить данную разметку. Весь морфологический разбор, кроме разметки именованных сущностей, никак не используется. Из-за того, что у меня нет мощностей для вычислений, приходилось обучаться не на всей выборке, а только на части. Модель показала очень хороший результат.

\medskip

\begin{tabular}{| l | l | l | l | l | l | l |}
\hline
Category               & Precision  &   Recall   &  F-score   &  Predicts  &   Golds    &  Correct   \\

\hline
 PER                                 & 99.768     & 90.727     & 95.033     & 2589       & 2847       & 2583       \\
\hline
\end{tabular}



*TODO: описание модели*

\subsubsection{BERT}

BERT есть для татарского языка, так что осталось его только запустить, что я ещё не сделала, но планирую вот уже на этой неделе. TODO: описать BERT.


\section{Воспроизведение статьи Невзоровой}

Была воспроизведена статья Невзоровой, на министерствах действительно показала хорошие результаты, но стало очевидно, что это полуручная история, потому что мусор пришлось выкидывать в ручном режиме. Ну и не удалось воспроизвести запросы в Туган Тел, а поиск был возможен только по слову (фразе). С помощью этого результата хочется разметить википедию, на википедии обучиться, а потом попытаться протестировать на Туган Тел и сравнить результаты.

\section{Сравнение результатов}

В процессе.



    \section{Заключение}

Проведена большая хорошая работа и по её окончанию получены удовлетворительные результаты. Была поставлены задачи получить размеченный корпус и обученную модель, распознающую именованные сущности и сравнить полученные результаты с предыдущими работами в данной области. Задачи были выполнена в полном объёме. Были размечены данные с помощью воспроизведенного алгоритма Невзоровой, улучшены с помощью ручной доработки; небольшое количество данных размечено полностью вручную. Получен список именованных сущностей, которые можно использовать в дальнейших работах по данной теме. Были обучены модели BiLSTM-CRF и BERT, которые показали результаты не хуже, чем предыдущая работа в данной области. 

В дальнейшем будет сотрудничать с Академией наук Республики Татарстан и дальше двигать направление распознавания именованных сущностей, пробовать новые модели не только распознавания, но также и разметки данных, поскольку с каждым годом корпус Туган Тел становится объемнее. Использовать в качестве признаков морфологические параметры и не только. Направлений для работы много и это хорошее поле для дальнейших исследований. Конкретные направления, которые хотелось бы выделить:

\begin{enumerate}
\item Улучшать разметку данных с помощью анализа текущих слабостей алгоритма Невзоровой, добавлением справочников и эвристик; 
\item Разработать удобную систему для ручной разметки данных и привлекать к разметке больше людей, получая более точные результаты;
\item Попробовать другие лучшие текущие модели, конкретно попробовать сделать обёртку CRF над BERT, это скорее всего поможет решить проблему с I-тегами без B-тегов.
\end{enumerate}

Выражаю благодарность О. Невзоровой за содействие в работе и предоставлении корпуса Туган Тел.
    
    \addtocounter{section}{1}
    \addcontentsline{toc}{section}{\thesection \ Список литературы}
    \bibliographystyle{gost2008}
    \bibliography{biblio}

\end{document}
