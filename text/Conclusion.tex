\section{Заключение}

Были поставлены задачи получить размеченный корпус и обученную модель, распознающую именованные сущности и сравнить полученные результаты с предыдущими работами в данной области. Задачи были выполнены в полном объёме. Были размечены данные с помощью воспроизведенного алгоритма Невзоровой, улучшены с помощью ручной доработки; небольшое количество данных размечено полностью вручную. Получен список n-грамм именованных сущностей, которые можно использовать в дальнейших работах по данной теме. Были обучены модели BiLSTM-CRF и BERT, которые показали результаты, сравнимые с предыдущей работой в данной области. 

В дальнейшем можно будет сотрудничать с Академией наук Республики Татарстан и дальше продвигать направление извлечения именованных сущностей, пробовать новые модели не только извлечения, но также и разметки данных, поскольку с каждым годом корпус Туган Тел становится объемнее. Использовать в качестве признаков морфологические параметры и не только. Направлений для работы много и это хорошее поле для дальнейших исследований. Конкретные направления, которые хотелось бы выделить:

\begin{enumerate}
\item Улучшать разметку данных с помощью анализа текущих слабостей алгоритма Невзоровой, добавлением справочников и эвристик; 
\item Разработать удобную систему для ручной разметки данных и привлекать к разметке больше людей, получая более точные результаты;
\item Попробовать другие лучшие известные модели, а именно сделать обёртку CRF над BERT, это скорее всего поможет решить проблему с I-тегами без B-тегов.
\end{enumerate}

Код доступен по ссылке \href{https://github.com/ksemiya/NER\_in\_Tatar}{github.com/ksemiya/NER\_in\_Tatar}

Выражаю благодарность О. Невзоровой за содействие в работе и предоставлении корпуса Туган Тел.