\section{Заключение}

Проведена большая хорошая работа и по её окончанию получены удовлетворительные результаты. Была поставлены задачи получить размеченный корпус и обученную модель, распознающую именованные сущности и сравнить полученные результаты с предыдущими работами в данной области. Задачи были выполнена в полном объёме. Были размечены данные с помощью воспроизведенного алгоритма Невзоровой, улучшены с помощью ручной доработки; небольшое количество данных размечено полностью вручную. Получен список именованных сущностей, которые можно использовать в дальнейших работах по данной теме. Были обучены модели BiLSTM-CRF и BERT, которые показали результаты не хуже, чем предыдущая работа в данной области. 

В дальнейшем будет сотрудничать с Академией наук Республики Татарстан и дальше двигать направление распознавания именованных сущностей, пробовать новые модели не только распознавания, но также и разметки данных, поскольку с каждым годом корпус Туган Тел становится объемнее. Использовать в качестве признаков морфологические параметры и не только. Направлений для работы много и это хорошее поле для дальнейших исследований. Конкретные направления, которые хотелось бы выделить:

\begin{enumerate}
\item Улучшать разметку данных с помощью анализа текущих слабостей алгоритма Невзоровой, добавлением справочников и др.; 
\item Разработать удобную систему для ручной разметки данных и привлекать к разметке людей, получая более точные результаты;
\item Попробовать другие лучшие текущие модели, конкретно попробовать сделать обёртку CRF над BERT, это скорее всего поможет решить проблему с I-тегами без B-тегов.
\end{enumerate}

Выражаю благодарность О. Невзоровой за содействие в работе и предоставлении корпуса Туган Тел.