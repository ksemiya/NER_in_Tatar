\section{Введение}

Распознавание именованных сущностей (Named entity recognition, NER) это одна из задач обработки естественного языка; задача детектирования и классификации имен в тексте на несколько заранее определённых категорий, таких как, например, люди, места, организации и т.д. NER имеет множество применений в прикладных задачах, используется в автоматическом разделении на категории текстов, рекомендательных системах, системах извлечения информации. Как идея распознавание именованных сущностей была изобретена ещё в прошлом веке, однако широкое распространение и большую скорость развития получила только в последнем десятилетии. Быстроразвивающиеся глубокие нейронные сети дали значительный толчок развитию обработке естественных языков, и, как следствие, NER. Были изобретены более эффективные и точные модели, которые показывают хорошие результаты. Однако существуют и серьёзные проблемы, связанные с данной задачей. Во-первых, упомянутые выше модели с хорошими языками существуют только для широко распространённых языков, для которых имеются размеченные корпуса. Во-вторых, большинство из существующих решений требуют большого количества размеченных текстов. В-третьих, исследователи не тратят время на нераспространённые языки, потому что это не самая важная задача, которая, скорее всего, не сможет принести большой выгоды в дальнейшем из-за сравнительно небольшого числа носителей.

Однако развитие распознавание именованных сущностей для татарского языка может быть использовано для всей группы тюркских языков, таких как алтайский, башкирский, чувашский, карачаево-балканский. Это связано с тем, что все языки тюркской группы достаточно похожи между собой, как грамматически, так и лексически, как следствие, решение задачи для одного языка скорее всего будет иметь неплохие шансы и для других языков данной группы. *написать что-нибудь про то, что язык надо сохранять и вообще все языки важны, особенно язык на 5 миллионов человек*

Сейчас тема распознавания именованных сущностей для языков с малыми ресурсами непопулярна, однако на тему конкретно татарского языка существует работа [ссылка] исследователей из Академии наук Республики Татарстан. В дальнейшем я более подробно рассмотрю их работу в своем исследовании.

Одна из главных проблем NER для языков с малыми ресурсами это отсутствие размеченных корпусов; как следствие задача разметки является одной из подзадач моей работы. 

В работе использовалась модель BiLSTM-CRF от [ссылка].


Черновик, вырезать в дальнейшем:


Данная работа представляет собой распознавание именованных сущностей в языке с малыми ресурсами (конкретно, татарским языком). 
Распознавание именованных сущностей может использоваться во многих прикладных задачах, таких как таргетинг, рекомендация для
новостной ленты, приложения для почты. На текущий момент это первая работа по татарскому языку (подробнее об этом в обзоре 
литературы), поэтому её актуальность не вызывает сомнений: она может быть использована в индустрии для внедрения современных
технологий в приложения для носителей языка. Несмотря на то, что татарский язык считается языком с малыми ресурсами, его 
активно используют более 5 миллионов человек (TODO ссылка на источник), что показывает важность поднятой мной темы.

